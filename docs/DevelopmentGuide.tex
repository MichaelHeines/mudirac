\documentclass[]{report}

\usepackage{amsmath}

% Title Page
\title{\texttt{mudirac} - a Dirac equation solver for muonic atoms
	\newline
	\large Development guide}
\author{Simone Sturniolo}


\begin{document}
\maketitle

\chapter*{Introduction}

This document contains information and references on the algorithms and methods employed for the development of \texttt{mudirac}. It is meant as a technical document, to keep track of technique used and allow for modification or bug fixing in the future.

\chapter{Theory}

\section{The Dirac equation}

The basic solver makes use of the radial Dirac equation written in its coupled form, here written in atomic units:

\begin{align}\label{dirac_sys}
	Q'(r) &= \frac{k}{r}Q + \left(mc-\frac{E-V(r)}{c}\right)P \\
	P'(r) &= -\frac{k}{r}P + \left(mc+\frac{E-V(r)}{c}\right)Q
\end{align}

as found for example in \cite{weinb2008, silbar2010} and others. In this form, the final wavefunction can be reconstructed by combining it with the appropriate spherical harmonics:

\begin{align}
\left<\mathbf{r}|\psi_{k\mu}\right> = \begin{pmatrix}
\frac{P_k}{r}\left<\mathbf{\hat{r}}|k\mu\right> \\
i\frac{Q_k}{r}\left<\mathbf{\hat{r}}|-k\mu\right>
\end{pmatrix}
\end{align}

where it must be kept in mind that the quantum number $k$ is $k=-(l+1)$ when $s = -1/2$ and $k=l$ when $s=1/2$, and the spin spherical harmonics are:

\begin{equation}
\left|k\mu\right> = \sum_{s=\pm 1/2} c(l\frac{1}{2};\mu-s, s)\left|l,\mu-s\right>\Phi(s)
\end{equation}

with

\begin{align}
\left<\mathbf{\hat{r}}|l,\mu-s\right> = Y_{l,\mu-s}(\mathbf{\hat{r}}) \\
\Phi\left(\frac{1}{2}\right) = \begin{pmatrix}
1 \\
0
\end{pmatrix}
\qquad
\Phi\left(-\frac{1}{2}\right) = \begin{pmatrix}
0 \\
1
\end{pmatrix} \\
c(l\frac{1}{2};\mu-s, s) = \frac{(-1)^{l+\mu-1/2}}{(2j+1)^{1/2}}\begin{pmatrix}
l \quad 1/2 \quad j \\
m \quad s \quad \mu
\end{pmatrix}
\end{align}

with $Y_{lm}$ being the spherical harmonics and the $c(l\frac{1}{2};\mu-s, s)$ being related to the Clebsch-Gordan coefficients \cite{weinb2008}.\newline
Since this equation is one dimensional we don't need to worry about Q and P being complex-valued, as they will be always real. In general, P will be the `large' component, and in the limit of $c \rightarrow \infty$ we would have that $P^2(r) = r^2\Psi^2(r)$ where $\Psi$ is just the normal radial component of the Schr\"{o}dinger equation solution. A higher Q corresponds to a more `relativistic' behaviour.

\chapter{Numerical implementation}

\section{Integration of the Dirac equation system}

\subsection{Grid}

The radial Dirac equation as written in \ref{dirac_sys} constitutes a system of two coupled first order differential equations. We use the shooting method to solve this problem. This approach requires one to choose the expected eigenvalue $E$ as a parameter, choose suitable guesses for the values at the boundary, then integrate numerically the equation from both sides of the domain up to a meeting point. The parameter is then adjusted based on how well the solutions match until stability is reached.\newline
There are various practical issues involved in solving the equation this way. The first thing is the grid to use. Here we choose to use a logarithmic radial grid, defined as:

\begin{equation}
r = r_0e^x \qquad x = \log\left(\frac{r}{r_0}\right)
\end{equation}

In this format the equations become

\begin{align}\label{dirac_sys_log}
\frac{\partial Q}{dx} &= kQ + r(x)\left(mc-\frac{E-V(x)}{c}\right)P \\
\frac{\partial P}{dx} &= -kP + r(x)\left(mc+\frac{E-V(x)}{c}\right)Q
\end{align}

The advantage is that an evenly spaced grid in $x$ is now much denser for small $r$, namely, for a Coulomb-like potential, close to the nucleus, where the wavefunction will vary more quickly.

\subsection{Boundary conditions}
Another issue is that of the boundary conditions. We take our cues for that from \cite{silbar2010}. For $r \rightarrow \infty$, one can see how $V \rightarrow 0$ and the equations reduce to

\begin{align}\label{dirac_sys_inf}
Q' &\approx \left(mc-\frac{E}{c}\right)P \\
P' &\approx \left(mc+\frac{E}{c}\right)Q
\end{align}

leading to the conditions (defining $K = \sqrt{m^2c^2-E^2/c^2}$):

\begin{align}
P(r) &\approx e^{-Kr} \\
Q(r) &\approx -\frac{K}{mc+E/c}P(r)
\end{align}

whereas for $r \rightarrow 0$ we find different results depending on whether we use a `true' Coulomb potential (point charge) or one where the charge is assumed to have a finite spherical size. For a point charge we find:

\begin{align}\label{dirac_sys_zero_pc}
	Q' &\approx \frac{1}{r}\left(kQ- \frac{Z}{c}P\right)  \\
	P' &\approx \frac{1}{r}\left(-kP + \frac{Z}{c}Q\right)
\end{align}

leading to

\begin{align}
P(r) &\approx r^{l+1} \\
Q(r) &\approx -\frac{Z}{c(l+2-k)}P(r)
\end{align}

while for a finite charge it's much simpler, as the equations decouple:

\begin{align}\label{dirac_sys_zero_sph}
Q' &\approx \frac{k}{r}Q  \\
P' &\approx -\frac{k}{r}P
\end{align}

so 

\begin{align}
P(r) &\approx r^{-k} \\
Q(r) &\approx r^k
\end{align}

\subsection{Potential}

The electrostatic potential for the equation is built out of two parts:

\begin{equation}
	V(r) = V_{N}+V_{bkg}
\end{equation}

Here $V_N$ is the Coulomb potential due to the nucleus, whereas $V_{bkg}$ is the contribution due to any other background charge (for example, the electronic cloud for muons). \newline
For a point-like nuclear charge, $V_N$ is simply:

\begin{equation}
	V_N(r) = -\frac{Z}{r}
\end{equation}

It should be noted that since we're dealing with negative muons, all potential terms are implicit multiplied by one negative unit of charge, which is the cause of the minus sign. In case one considers a finite sized spherical nucleus of uniform charge and radius $R$, one finds:

\begin{equation}
V_N(r) = \begin{cases}
-\frac{Z}{R}\left(\frac{3}{2}-\frac{1}{2}\frac{r^2}{R^2}\right), & \text{$r\leq R$}.\\
-\frac{Z}{r}, & \text{$r > E$}.
\end{cases}
\end{equation}

For the background charge, one builds the potential by integrating the equation:

\begin{equation}
\nabla^2 V = \rho
\end{equation}

which in radial coordinates becomes

\begin{equation}
	\frac{\partial}{\partial r}\left(r^2	\frac{\partial V}{\partial r}\right) = 4\pi r^2 \rho(r)
\end{equation}

where $4\pi r^2 \rho(r)$ represents the charge density of a spherical shell. Since we're using a logarithmic grid, when expressing the derivatives with respect to the $\log(r)$ we have

\begin{equation}
\frac{\partial^2 V}{\partial x^2} + \frac{\partial V}{\partial x}= 4\pi r^2 \rho(r)
\end{equation}

Due to the presence of a first order derivative this equation can not be integrated using Numerov's method. Instead, a finite order derivative scheme was used making use of the following approximations:

\begin{align}
\left.\frac{\partial V}{\partial x}\right|_i &\approx \frac{1}{h}\left(\frac{11}{6}V_i -3V_{i-1} +\frac{3}{2}V_{i-2}-\frac{1}{3}V_{i-3} \right) \\
\left.\frac{\partial^2 V}{\partial x^2}\right|_i &\approx \frac{1}{h^2}\left(2V_i -5V_{i-1} +4V_{i-2}-V_{i-3} \right) \\
\end{align}

setting the first three points of V with an approximation of constant charge for $r < r_0$ and shooting outwards for all the following ones.

\subsection{Integration scheme}

In order to integrate numerically the equations we make use of the following second order scheme for the first derivative of a function defined on a uniform grid with step $h$:

\begin{equation}
y'_i = \frac{3y_i-4y_{i-1}+y_{i-2}}{2h} + \mathcal{O}(h^3)
\end{equation}

and its backwards converse:

\begin{equation}
y'_i = -\frac{3y_i-4y_{i+1}+y_{i+2}}{2h} + \mathcal{O}(h^3)
\end{equation}

With this definition, any coupled system of the form:

\begin{align}
Q'_i &= AA_i Q_i + AB_iP_i \\
P'_i &= BA_i Q_i + BB_iP_i
\end{align}

can be solved by seeing how it's

\begin{align}
Q_i\left(\pm\frac{3}{2h}-AA_i\right) - AB_iP_i&= \pm\frac{1}{h}\left(2Q_{i\mp1}-\frac{1}{2}Q_{i\mp2}\right) \\
P_i\left(\pm\frac{3}{2h}-BB_i\right) - BA_iQ_i&= \pm\frac{1}{h}\left(2P_{i\mp1}-\frac{1}{2}P_{i\mp2}\right)
\end{align}

the signs corresponding respectively to forward and backwards integration. Solving this linear system leads to the final formulas for $Q_i$ and $P_i$ at each step. If we define

\begin{align}
AC_i &= \pm\frac{1}{h}\left(2Q_{i\mp1}-\frac{1}{2}Q_{i\mp2}\right) \\
BC_i &= \pm\frac{1}{h}\left(2P_{i\mp1}-\frac{1}{2}P_{i\mp2}\right) \\
\end{align}

then we have

\begin{align}
Q_i &= \frac{-BC_iAB_i-AC_i(\pm\frac{3}{2h}-BB_i)}{AB_iBA_i-(\pm\frac{3}{2h}-AA_i)(\pm\frac{3}{2h}-BB_i)} \\
P_i &= \frac{-AC_iBA_i-BC_i(\pm\frac{3}{2h}-AA_i)}{AB_iBA_i-(\pm\frac{3}{2h}-AA_i)(\pm\frac{3}{2h}-BB_i)} \\
\end{align}

and this allows one to progressively integrate the equations given two initial points set with the boundary conditions.

\subsection{Finding the eigenvalue}

The equations are integrated forward from $r\approx0$ and backwards from $r\approx\infty$ up to a meeting point. This is chosen to be the `turning point', namely the first point on the grid (going outwards) for which

\begin{equation}
E-mc^2 = B < V(r)
\end{equation}

where we call $B$ the `binding energy' of the particle, corresponding to the traditional energy eigenvalue in the Schr\"{o}dinger equation. This is the effective energy the particle has at its disposal to escape the potential, besides its rest mass energy term $mc^2$.\newline
Since the wavefunctions are not normalized yet, their values can all be scaled by a constant term. This means that comparing them directly is meaningless. Rather, in order to find whether the solutions `match', one can impose the condition:

\begin{equation}
\left. \frac{Q}{P}\right|_f = \left. \frac{Q}{P}\right|_b
\end{equation}

with f and b indicating forward or backwards integration. Since this is not usually the case, one will attempt a first value of $E$, then try to calculate 

\begin{equation}
\frac{\partial}{\partial E}\left(\left. \frac{Q}{P}\right|_f - \left. \frac{Q}{P}\right|_b\right) = 
\frac{\partial}{\partial E}\left. \frac{Q}{P}\right|_f - \frac{\partial}{\partial E}\left. \frac{Q}{P}\right|_b
\end{equation}

to then use a steepest descent algorithm to find the correct value of $E$. This tends to be a very well behaved function, so optimisation is straightforward. One approach to doing this is numerical - one simply integrates the equation both at $E$ and $E+\delta E$, then computes the derivative from the difference of the values found. However it's more efficient to look directly at the behaviour of $y = Q/P$ as its own function. By dividing $Q'$ by $P$ and defining for convenience:

\begin{equation}
g_\pm = \left(mc\pm\frac{E-V}{c}\right)
\end{equation}

we have

\begin{equation}
\frac{Q'}{P} = \frac{k}{r}\frac{Q}{P}+g_-
\end{equation}

and since (making use of the other equation for $P'$)

\begin{equation}
y' = \frac{Q'}{P} - \frac{Q}{P^2}P' = \frac{Q'}{P} + \frac{k}{r} y -g_+y^2
\end{equation}

we have

\begin{equation}
y' = 2\frac{k}{r}y-g_+y^2+g_-
\end{equation}

However what we are interested in is $\zeta = \partial y / \partial E$. We can derive and find:

\begin{equation}
\zeta' = 2\left(\frac{k}{r}-g_+y\right)\zeta-\frac{1}{c}(y^2+1)
\end{equation}

Since we already know the values of Q and P, and thus of y, integrating numerically this equation is trivial. We then have that

\begin{equation}
\frac{\partial}{\partial E}\left. \frac{Q}{P}\right|_f - \frac{\partial}{\partial E}\left. \frac{Q}{P}\right|_b = \zeta_f-\zeta_b
\end{equation}


\bibliographystyle{unsrt}
\bibliography{DevelopmentGuide.bib}

\end{document}          
